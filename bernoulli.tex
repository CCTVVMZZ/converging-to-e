% -*- mode: latex; eval: (flyspell-mode 1); ispell-local-dictionary: "american"; TeX-master: "converge"; -*-

\newcommand{\ttx}{\mathtt{x}}

%As promised, let us return to Bernoulli's inequality.
Let  $\ttx$ denote an indeterminate over the real numbers.
Put
$$
f_n(\ttx) = {(1 + \ttx)}^n - 1 - n \ttx 
$$
for each $n \in \bN$.
Lemma~\ref{lem:Bernoulli} states that $f_n(- x)$ is non-negative for every $n \in \bN$ and every $x \in [0, 1]$.

\begin{lemma} \label{lem:root-mult}
  For each integer $n \ge 2$,
  $0$ is a double root of $f_n$ and $0$ is the only non-simple (complex) root of $f_n$.
\end{lemma}

\begin{proof}
  Straightforward computations yield
  \begin{equation} \label{eq:deriv-fn} 
  f_n'(\ttx)  = n \left( {(1 + \ttx)}^{n - 1} -  1 \right) 
  \end{equation}
  and
  $$
  f_n''(\ttx)  = n (n - 1) {(1 + \ttx)}^{n - 2} \,.
  $$
  It follows
  $$
  f_n(0) = f_n'(0) = 0 \ne n (n - 1) = f_n''(0) \, ,
  $$
  whence $0$ is a double root of $f_n$.
  In addition, $f_n$ and its derivative satisfy 
   $$
   (1 + \ttx) f_n'(\ttx) - n f_n(\ttx) = n (n - 1) \ttx \, ,
   $$
   so for every complex number $z$, $f_n(z) = f_n'(z) = 0$ implies $z = 0$.
   It follows that $0$ is the only non-simple root of $f_n$.
 \end{proof}

 The following identities hold for every $n \in \bN$ and may be of interest to the reader:
 $$
 f_n (\ttx)
 = \ttx^2 \sum_{k = 2}^n \binom{n}{k} \ttx^{k - 2}
 = \ttx^2 \sum_{k = 1}^{n - 1}  \sum_{j = 0}^{k - 1} {(1 + \ttx)}^j \, .  
 $$

 % $$
 % {(1 + \ttx)}^n - 1 = \ttx \sum_{k = 0}^{n - 1} {(1 + \ttx)}^k 
 % $$

\begin{lemma} \label{lem:even-one-root}
  For each even integer $n \ge 2$,
  $0$ is the only real root of $f_n$.
 \end{lemma} 

 \begin{proof}
   Since $n - 1$ is positive and odd,
   $1$ is the only real $(n - 1)$th root of unity, 
   and thus Equation~\eqref{eq:deriv-fn} shows that $0$ is the only real root of $f_n'$.
Therefore, it follows from Rolle's theorem that $0$ is the only real root of $f_n$.
 \end{proof} 

 \begin{lemma} \label{lem:odd-two-roots}   
   For each odd integer $n \ge 3$,
   exactly one root of $f_n$ is both real and non-zero and that root is less than $- 2$.
 \end{lemma}

 \begin{proof}
   Since $n - 1$ is positive and even,
   $- 1$ and $1$ are the only real $(n - 1)$th roots of unity,
   and thus Equation~\eqref{eq:deriv-fn} shows that $- 2$ and $0$ are the only the real roots of $f_n'$.
   Therefore, it follows from Rolle's theorem that
   $f_n$ has no root in $\left[- 2, 0 \right[ \cup \left]0,  \infty \right[$
   and that
   at most one root of $f_n$ lies in $\left]- \infty, - 2 \right[$.
   On the other hand, we have $f_n(-1) = n - 1 > 0$ and $f_n(x)$ approaches $- \infty$ as $x$ approaches $- \infty$,
   so the intermediate value theorem ensures that at least one root of $f_n$ lies in $\left]- \infty, -1 \right[$.
 \end{proof}

 \begin{theorem}[Bernoulli's inequality] \label{thm:Bernoulli}
   Let $n$ be an integer greater than $1$ and let $x$ be a non-zero real number.
   If $n$ is even or if $x \ge - 2$ then $f_n(x)$ is positive.
 \end{theorem}
 
 \begin{proof}
   %Let $n$ be a fixed integer greater than~$1$.
   Put $I = \left[- 2, \infty \right[$.
   It follows from Lemmas \ref{lem:even-one-root} and  \ref{lem:odd-two-roots}
   that $0$ is the only root of $f_n$ that lies in~$I$.
   Therefore, 
   Besides, $I$ is an interval and Lemma~\ref{lem:root-mult} ensures that $0$ is a double root of $f_n$.
   Therefore, $f_n$ does not change sign on $I \setminus \{ 0 \}$
   (the graph of $f_n$ does \emph{not cross}  the $x$-axis at $x = 0$).
   Since $f_n$ approaches $\infty$ as $x$ approaches $\infty$,
   $f_n$ is positive on $I \setminus \{ 0 \}$.
   It remains to prove that $f_n$ is positive on $\left]- \infty, - 2\right[$ if $n$ is even.
   Assume that $n$ is even.
   Lemmas~\ref{lem:even-one-root} ensures that $f_n$ does not change sign on
 \end{proof}
 
 \begin{proof}
   First, Lemmas \ref{lem:even-one-root} and  \ref{lem:odd-two-roots} ensure that
   $f_n$ has not root in  $\left]0, \infty \right[$.   
   Besides, $f_n(t)$ approaches $\infty$ as $t$ approaches $\infty$.
   Therefore, $f_n$ is positive on $\left]0, \infty \right[$.
   Second, assume that $n$ is even.
   Then, Lemma~\ref{lem:even-one-root} ensures that $f_n$ has no negative roots.
   Besides,  $f_n(t)$ approaches $\infty$ as $t$ approaches $- \infty$.
   Therefore, $f_n$ is positive on $\left]- \infty, 0 \right[$.
   Third, assume that $n$ is odd.
   Lemma~\ref{lem:even-one-root} ensures that $f_n$ has no root in $\left[- 2, 0 \right[$.
   Besides, $f(- 1) = n - 1$ is positive.
   Therefore, $f_n$ is positive on $\left]- \infty, 0 \right[$.
 \end{proof}
 

 \begin{theorem}
   For each odd integer $n \ge 3$,
   the non-zero real root of $f_n$ is not greater than $- 2 - {(n - 2)}^{- 1}$.
 \end{theorem}

 \begin{proof}
   It suffice to check that for each integer $n \ge 3$,
    $f_n$ is non-negative on $\left[a_n, \infty \right[$,
    where $a_n = - 2 - {(n - 2)}^{- 1}$.
    Since $f_3(- 3) = 0$ the basis of our induction holds true.
    Straightforward computations yield 
    $$
    f_{n + 2} (\ttx) = {(1 + \ttx)}^2 f_n(\ttx) + n \ttx^2 (\ttx - a_{n + 2}) 
    $$
   
 \end{proof} 
 
 

