% -*- mode: latex; eval: (flyspell-mode 1); ispell-local-dictionary: "american"; TeX-master: t; -*-

\documentclass[12pt,a4paper]{article}



\usepackage{hyperref,amsthm,amsmath,amsfonts} 

\newcommand{\bN}{\mathbb{N}} 
\newcommand{\Rnneg}{\left[0, \infty\right[}
\newcommand{\Rpos}{\left]0, \infty\right[}

\newtheorem{theorem}{Theorem}
\newtheorem{lemma}{Lemma}

\begin{document}

\sloppy

\title{On the behavior of $n \mathbin{/} \sqrt[n]{n!}$ as $n$ approaches infinity} 
\author{Charlot Colmes}
\maketitle

Let $\bN$ denote set of all \emph{positive} integers and
let Euler's number be defined as:
$$
e = \sum_{k = 0}^\infty \frac{1}{k!} \,. 
$$
The aim of this paper is to present a simple proof for the following theorem:

\begin{theorem} \label{thm:factnn-e}
If
$$
u_n = \frac{n}{\sqrt[n]{n!}}
$$
for every $n \in \bN$ then
the sequence $\left( u_n  \right)_{n \in \bN}$ is increasing and converges to~$e$.
\end{theorem}

The convergence to $e$ can easily be deduced from \emph{Stirling's formula}
\cite{GiaquintaModicaApprox, RudinPrinciples}
but using such a tool is clearly an overkill.
Our proof of Theorem~\ref{thm:factnn-e} is self-contained and 
does not rely on any differential or integral calculus.

As an introduction for our first lemma,
recall that for every $\left( u_n \right)_{n \in \bN} \in \Rpos^\bN$, we have 
$$
\liminf_{n \to \infty} \frac{u_{n + 1}^{n + 1}}{ u_n^{n}} 
\le
\liminf_{n \to \infty} u_n 
\le
\limsup_{n \to \infty} u_n
\le 
\limsup_{n \to \infty} \frac{u_{n + 1}^{n + 1}}{u_n^{n}} 
$$
\cite{RudinPrinciples},
so if $\left( u_{n + 1}^{n + 1} u_n^{- n} \right)_{n \in \bN}$ converges then
$\left( u_n \right)_{n \in \bN}$ converges to the same limit.
In particular, every series that passes the \emph{ratio test} also passes the \emph{root test}.
The latter results are closely related to:

\begin{lemma} \label{lem:root-vs-ratio}
  Let $\left( u_n \right)_{n \in \bN}  \in \Rpos^{\bN}$ be such that
  $u_1 < u_2$ and  
  $\left(  u_{n + 1}^{n + 1}  u_n^{-n} \right)_{n \in \bN}$ is monotonically increasing.
  Then,
  $\left(u_n \right)_{n \in \bN}$ is increasing and
  converges to the same limit as $\left(  u_{n + 1}^{n + 1}  u_n^{-n} \right)_{n \in \bN}$.
\end{lemma}

\begin{proof}
 Put
  $v_n = u_{n + 1} u_n^{-1}$
  and
  $w_n = u_{n + 1}^{n + 1} u_n^{- n}$
  for every $n \in \bN$.
  
  Let us first prove that $\left( u_n \right)_{n \in \bN}$ is increasing.
  More precisely, let us prove by induction on $n$ that $v_n$ is greater than $1$ for every $n \in \bN$.
  Since $v_1$ is greater than $1$ by assumption,
  the basis of our induction holds true.
  Now, let $n \in \bN$. %be such that $v_n > 1$.
  Straightforward computations yield
  \begin{equation} \label{eq:wn-uv}
    w_n = u_{n + 1}v_n^n \,, 
  \end{equation} 
  and subsequently, 
  $w_{n + 1} = u_{n + 1} v_{n + 1}^{n + 2}$.
  It follows  
  $v_{n + 1}^{n + 2}= w_{n + 1}w_n^{-1} v_n^n \ge v_n^n$,
  and thus $v_n > 1$ implies $v_{n + 1} > 1$, as desired.

  It remains to prove $\bar u = \bar w$, where 
  $\bar u = \lim_{n \to \infty} u_n$ and 
  $\bar w = \lim_{n \to \infty} w_n$.
  Let $n \in \bN$.
  Since $v_n > 1$,
  Equation~\eqref{eq:wn-uv} yields $w_n > u_{n + 1}$,
  and thus we obtain $\bar w \ge \bar u$ by letting $n$ approach~$\infty$.
  In particular, $\bar u = \infty$ implies $\bar w = \infty$.
  Hence, the final step is to prove $\bar u \ge \bar w$ under the assumption $\bar u \ne \infty$.
  Equality 
  $$
  \frac{u_{2n}^{2n}}{ u_n^n} =  \prod_{k = n}^{2n - 1} w_k
  $$
  holds true because the latter product is telescoping.
  Moreover, each factor $w_k$ of that product is larger than or equal to $w_n$.
  It follows 
  $u_{2n}^{2n} u_n^{-n} \ge w_n^n$ 
  or, equivalently,
  $u_{2n}^2u_n^{- 1}  \ge  w_n$.
  Since $0 < \bar u < \infty$, 
  $u_{2n}^2 u_n^{-1}$ approaches $\bar u$ as $n$ approaches $\infty$.
  It follows $\bar u \ge \bar w$, as desired.
\end{proof}

As an aside, let us check that the converse of Lemma~\ref{lem:root-vs-ratio} is false.
Set $ u_n = \exp \left( - n^{- 2} \right)$ for each $n \in \bN$.
Clearly,  $\left( u_n \right)_{n \in \bN}$  is increasing whereas
$\left(  u_{n + 1}^{n + 1}  u_n^{-n} \right)_{n \in \bN}$ is decreasing because 
$$
\frac{u_{n + 1}^{n + 1}}{u_n^n} = \exp \left(\frac{1}{n (n + 1)} \right) 
$$
for every $n \in \bN$.

\begin{lemma} \label{lem:Bernoulli}
   Inequality ${(1 - x)}^n \ge  1 - n x$ 
   holds true for every $x \in [0, 1]$ and every $n \in \bN$.
   % Inequality ${(1 - x)}^n > 1 - n x $
   % holds true for every $x \in \left]0, 1 \right]$ and every integer $n \ge 2$.
\end{lemma} 

\begin{proof}
  Let $x \in [0, 1]$ be fixed.
  Put
  $$
  u _n =  \left( 1 - x \right)^n + n x  
  $$
  for each $n \in \bN$.
  Straightforward computations yield
  $$
  u_{n + 1} - u_n = \left(1 - {(1 - x)}^n \right) x \ge 0
  $$
  for every $n \in \bN$,
  so the sequence $\left( u_n \right)_{n \in \bN}$ is monotonically increasing.
  In particular, we have $u_n \ge u_1 = 1$ for every $n \in \bN$,
  and thus the desired inequality holds true.  
\end{proof}


Lemma~\ref{lem:Bernoulli} states a very special case of \emph{Bernoulli's inequality} \cite{MitrinovicAI}.

\begin{lemma} \label{lem:convergence-to-e}
  If
  $$
  u_n = \left( 1 + \frac{1}{n} \right)^n
  $$
  for every $n \in \bN$ then the sequence $\left( u_n  \right)_{n \in \bN}$ is increasing and converges to~$e$.
\end{lemma}

\begin{proof}
  Let us first check that $\left( u_n \right)_{n \in \bN}$ is increasing.
  %For each $n \in \bN$, put  $v_n = 1 + n^{-1}$: $u_n = v_n^n$.
  %Put  $v_n = 1 + n^{-1}$ for each $n \in \bN$. 
  Let $n \in \bN$.
  Our task is to prove $u_{n + 1} u_n^{-1} > 1$.
  Straightforward computations yield
  \begin{align*}
    \frac{1}{u_n} & = \left( \frac{n}{n + 1}  \right)^n \,, \\
    u_{n + 1} & = \left( \frac{n + 2}{n + 1} \right)^n
                \frac{n + 2}{n + 1} \,,
  \end{align*}
  and 
  $$
  \frac{n}{n + 1}
  \cdot 
  \frac{n + 2}{n + 1}
  = 1 - \frac{1}{{(n + 1)}^2}  \,,
   $$
  whence 
  $$
  \frac{u_{n + 1}}{u_n}
  =
  \frac{1}{u_n} \cdot u_{n + 1}  
  =
  \left(
  \frac{n}{n + 1} \cdot \frac{n + 2}{n + 1}
  \right)^n
  \frac{n + 2}{n + 1}
  =
  \left( 1 - \frac{1}{{(n + 1)}^2} \right)^n 
  \frac{n + 2}{n + 1}\, .
  $$
  Besides, we get  
   $$
   \left( 1 - \frac{1}{{(n + 1)}^2}  \right)^n \ge  1 - \frac{n}{{(n + 1)}^2} 
 $$
 by letting $x = {(n + 1)}^{-2}$ in Lemma~\ref{lem:Bernoulli}.
 It follows
 $$
 \frac{u_{n + 1}} {u_n}
 \ge
 \left( 1 - \frac{n}{{(n + 1)}^2} \right)
  \frac{n + 2}{n + 1}
 =
 1 + \frac{1}{{(n + 1)}^3}
 > 1 \,, 
 $$
 as desired.
  
  It remains to prove $\bar u = e$, where $\bar u = \lim_{n \to \infty} u_n$.
  Put
  $$
  v_n = \sum_{k = 0}^n \frac{1}{k!} 
  $$
  for each $n \in \bN$.
  Let $m$, $n \in \bN$ be such that $m \le n$.
  The binomial theorem yields
  $$
  u_n
  = \sum_{k = 0}^n  \binom{n}{k} \frac{1}{n^k}
  = \sum_{k = 0}^n \frac{1}{k!} \cdot \frac{n!}{n^k {(n - k)}!} \, .
  $$
  Besides, $n!$ is less than or equal to $n^k {(n - k)}!$ for each $k \in \{ 0, 1, 2, \dotsc, n \}$.
  It follows $ u_n \le v_n$,  and consequently, $\bar u \le e$.
  The final step is to prove the other inequality.
 For each fixed $k \in \{ 0, 1, 2, \dotsc, n \}$, 
  $$
  \frac{n!}{{n^k (n - k)}!} = \prod_{j = 0}^{k - 1} \left( 1 - \frac{j}{n} \right) 
 $$
 approaches $1$ as $n$ approaches~$\infty$.
 Therefore, by letting $m$ be fixed and $n$ approach $\infty$ in 
 $$
  \sum_{k = 0}^m \frac{1}{k!} \cdot \frac{n!}{n^k {(n - k)}!} \le u_n \,, 
  $$
  we obtain $v_m \le \bar u$, and consequently, $e \le \bar u$.
\end{proof}


Let us briefly comment the proof of Lemma~\ref{lem:convergence-to-e}.
Our proof of the fact that $\left( \left( 1 + n^{-1} \right)^n  \right)_{n \in \bN}$
is increasing is taken from \cite{Wiener85}.
Another approach \cite{GiaquintaModicaApprox} is to closely examine the right-hand side of equality 
$$ 
\left(1 + \frac{1}{n} \right)^n = \sum_{k = 0}^n \frac{1}{k!} \prod_{j = 0}^{k - 1} \left(1 - \frac{j}{n} \right) 
$$
which holds true for every $n \in \bN$.
% Equation~\eqref{eq:Bernoulli} is usually called \emph{Bernoulli's inequality};
% it holds true for every $n \in \bN$ and every
% $x \in \left[- 1, 1 \right[ \cup \left]1, \infty \right[$ 
Our proof of the fact that
$\left( \left( 1 + n^{-1} \right)^n  \right)_{n \in \bN}$ converges to $e$ is taken from \cite{RudinPrinciples}.


\begin{proof}[Proof of Theorem~\ref{thm:factnn-e}]
  Straightforward computations yield
  $$
  \frac{u_{n + 1}^{n + 1}}{u_n^n} 
  = \left(1 + \frac{1}{n} \right)^n  
  $$
  for every $n \in \bN$,
  so Lemma~\ref{lem:convergence-to-e} ensures that 
  $\left( u_{n + 1}^{n + 1} u_n^{- n} \right)_{n \in \bN}$ is (monotonically) increasing and converges to~$e$.
  Besides, we have $u_1 = 1 < \sqrt{2} = u_2$.
  Therefore, the desired result follows from Lemma~\ref{lem:root-vs-ratio}.
\end{proof} 


\bibliographystyle{plain}
\bibliography{converge}

\end{document}
  
  