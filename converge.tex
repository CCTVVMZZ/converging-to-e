% -*- mode: latex; eval: (flyspell-mode 1); ispell-local-dictionary: "american"; TeX-master: t; -*-

\documentclass[12pt]{article}



\usepackage{hyperref,amsthm,amsmath,amsfonts,enumerate,stmaryrd}

\newcommand{\bN}{\mathbb{N}} %{\bZ_{\ge 0}}
\newcommand{\Rnneg}{\left[0, \infty\right[} %{\bR_{\ge 0}}
\newcommand{\Rpos}{\left]0, \infty\right[} %{\bR_{\ge 0}}
\newcommand{\Rneg}{\left]- \infty, 0 \right[} %{\bR_{\ge 0}}
\newcommand{\Rnpos}{\left]- \infty, 0 \right]} %{\bR_{\ge 0}}
\newcommand{\abs}[1]{\left| #1 \right|}

\newtheorem{theorem}{Theorem}
\newtheorem{lemma}{Lemma}

\begin{document}

\sloppy

\title{On the behavior of $n \mathbin{/} \sqrt[n]{n!}$ as $n$ approaches infinity} 
\author{Charlot Colmes}
\maketitle

Let $\bN$ denote set of all \emph{positive} integers and
let Euler's number be defined as:
$$
e = \sum_{k = 0}^\infty \frac{1}{k!} \,. 
$$
The aim of this paper is to present a simple proof for the following theorem:

\begin{theorem} \label{thm:factnn-e}
If
$$
u_n = \frac{n}{\sqrt[n]{n!}}
$$
for every $n \in \bN$ then
the sequence $\left( u_n  \right)_{n \in \bN}$ is increasing and converges to~$e$.
\end{theorem}

The convergence to $e$ can easily be deduced from \emph{Stirling's formula}
\cite{GiaquintaModicaApprox, RudinPrinciples}
but using such a tool is clearly an overkill.
Our proof of Theorem~\ref{thm:factnn-e} is self-contained.
In particular, it does not rely on any differential or integral calculus.

As an introduction for our first lemma,
recall that for every $\left( u_n \right)_{n \in \bN} \in \Rpos^\bN$, we have 
$$
\liminf_{n \to \infty} \frac{u_{n + 1}^{n + 1}}{ u_n^{n}} 
\le
\liminf_{n \to \infty} u_n 
\le
\limsup_{n \to \infty} u_n
\le 
\limsup_{n \to \infty} \frac{u_{n + 1}^{n + 1}}{u_n^{n}} 
$$
\cite{RudinPrinciples},
so if $\left( u_{n + 1}^{n + 1} u_n^{- n} \right)_{n \in \bN}$ converges then
$\left( u_n \right)_{n \in \bN}$ converges to the same limit.
In particular, every series that passes the \emph{ratio test} also passes the \emph{root test}.
The latter results are closely related to:

\begin{lemma} \label{lem:root-vs-ratio}
  Let $\left( u_n \right)_{n \in \bN}  \in \Rpos^{\bN}$ be such that
  $u_1 < u_2$ and  
  $\left(  u_{n + 1}^{n + 1}  u_n^{-n} \right)_{n \in \bN}$ is monotonically increasing.
  Then,
  $\left(u_n \right)_{n \in \bN}$ is increasing and
  converges to the same limit as $\left(  u_{n + 1}^{n + 1}  u_n^{-n} \right)_{n \in \bN}$.
\end{lemma}

\begin{proof}
 Put
  $v_n = u_{n + 1} u_n^{-1}$
  and
  $w_n = u_{n + 1}^{n + 1} u_n^{- n}$
  for every $n \in \bN$.
  
  Let us first prove that $\left( u_n \right)_{n \in \bN}$ is increasing.
  More precisely, let us prove by induction on $n$ that $v_n$ is greater than $1$ for every $n \in \bN$.
  Since $v_1$ is greater than $1$ by assumption,
  the basis of our induction holds true.
  Now, let $n \in \bN$. %be such that $v_n > 1$.
  Straightforward computations yield
  \begin{equation} \label{eq:wn-uv}
    w_n = u_{n + 1}v_n^n \,, 
  \end{equation} 
  and subsequently, 
  $w_{n + 1} = u_{n + 1} v_{n + 1}^{n + 2}$.
  It follows  
  $v_{n + 1}^{n + 2}= w_{n + 1}w_n^{-1} v_n^n \ge v_n^n$,
  and thus $v_n > 1$ implies $v_{n + 1} > 1$, as desired.

  It remains to prove $\bar u = \bar w$, where 
  $\bar u = \lim_{n \to \infty} u_n$ and 
  $\bar w = \lim_{n \to \infty} w_n$.
  Let $n \in \bN$.
  Since $v_n > 1$,
  Equation~\eqref{eq:wn-uv} yields $w_n > u_{n + 1}$,
  and thus we obtain $\bar w \ge \bar u$ by letting $n$ approach~$\infty$.
  In particular, $\bar u = \infty$ implies $\bar w = \infty$.
  Hence, the final step is to prove $\bar u \ge \bar w$ under the assumption $\bar u \ne \infty$.
  Equality 
  $$
  \frac{u_{2n}^{2n}}{ u_n^n} =  \prod_{k = n}^{2n - 1} w_k
  $$
  holds true because the latter product is telescoping.
  Moreover, each factor $w_k$ of that product is larger than or equal to $w_n$.
  It follows 
  $u_{2n}^{2n} u_n^{-n} \ge w_n^n$ 
  or, equivalently,
  $u_{2n}^2u_n^{- 1}  \ge  w_n$.
  Since $\bar u \notin \{ 0, \infty \}$, 
  $u_{2n}^2 u_n^{-1}$ approaches $\bar u$ as $n$ approaches $\infty$.
  It follows $\bar u \ge \bar w$, as desired.
\end{proof}

As an aside, let us check that the converse of Lemma~\ref{lem:root-vs-ratio} is false.
Set $ u_n = \exp \left( - n^{- 2} \right)$ for each $n \in \bN$.
Clearly,  $\left( u_n \right)_{n \in \bN}$  is increasing whereas
$\left(  u_{n + 1}^{n + 1}  u_n^{-n} \right)_{n \in \bN}$ is decreasing because 
$$
\frac{u_{n + 1}^{n + 1}}{u_n^n} = \exp \left(\frac{1}{n (n + 1)} \right) 
$$
for every $n \in \bN$.

\begin{lemma}
  Inequality 
  \begin{equation} \label{eq:Bernoulli}
    \prod_{n = 1}^\infty  (1 - x_n)  \ge 1  + x_1 x_2 - \sum_{n = 0}^\infty x_n 
  \end{equation}
  holds true for every $\left( x_n \right)_{n \in \bN} \in {[0, 1]}^\bN$.
 % holds true if, and only if, there exist $i$, $j \in \bN$ such that $i \ne j$ and $x_i x_j \ne 0$.
  % For each $\left( x_n \right)_{n \in \bN} \in {[0, 1]}^\bN$,
  % inequality 
  % \begin{equation} \label{eq:Bernoulli}
  %   \prod_{n = 1}^\infty  (1 - x_n)  > 1  + x_1 x_2 - \sum_{n = 0}^\infty x_n 
  % \end{equation} 
  % holds true if, and only if, there exist $i$, $j \in \bN$ such that $i \ne j$ and $x_i x_j \ne 0$.
\end{lemma} 

\begin{proof}
  Put
  \begin{align*}
  s_n  & = \sum_{k = 1}^n x_k \,, \\
  p_n &  = \prod_{k = 1}^n (1 - x_k) \,, 
  \end{align*}
  and $d_n = s_n + p_n$ for each $n \in \bN \cup \{ \infty \}$.
  Since
  $s_{n + 1} - s_n = x_{n + 1} \ge 0$
  for every $n \in \bN$,
  the sequence $\left( s_n \right)_{n \in \bN}$ is monotonically increasing and converges to $s_\infty$;
  since
  $p_n - p_{n + 1} =  p_n x_{n + 1} \ge 0$
  for every $n \in \bN$, 
  the sequence $\left( p_n \right)_{n \in \bN}$ is monotonically decreasing and converges to $p_\infty$.
  For each $n \in \bN$,
  straightforward computations yield
  \begin{align*}
    d_{n + 1} - d_n 
  & = (s_{n + 1} - s_n) - (p_n - p_{n + 1}) \\
  & = x_{n + 1} - p_n x_{n + 1}  \\
    & =  (1 - p_n) x_{n + 1} \\
    & \ge 0   \,,  
  \end{align*}
  so the sequence $\left( d_n  \right)_{n \in \bN}$ is monotonically increasing and converges to $d_\infty$.
  Hence, we have $d_\infty \ge d_2 =  1 + x_1 x_2$, as desired.  
\end{proof}


\begin{lemma} \label{lem:convergence-to-e}
  If
  $$
  u_n = \left( 1 + \frac{1}{n} \right)^n
  $$
  for every $n \in \bN$ then the sequence $\left( u_n  \right)_{n \in \bN}$ is increasing and converges to~$e$.
\end{lemma}

\begin{proof}
  Let us first prove that $\left( u_n \right)_{n \in \bN}$ is increasing.
  Let $n \in \bN$.
  
  and let $x \in \left[- 1, 1 \right[$.
  % Since $x^k \le 1$  for every $k \in \bN$,
  % we have     
  % $$
  % \frac{x^{n + 1} - 1}{x - 1} = \sum_{k = 0}^n x^k \le n + x < n + 1 \,, 
  % $$
  % and consequently, 
  % \begin{equation} 
  % x^{n + 1} > 1 + (n + 1)(x - 1) \,.
  % \end{equation}
  % Now, set
  % $$
  % x = \frac{1 + {(n + 1)}^{-1}} {1 + n^{-1}} \,, 
  % $$
  % and then multiply Equation~\eqref{eq:Bernoulli} by $1 + n^{-1}$ on both sides.
  % We obtain $u_{n + 1} u_n^{-1} > 1$, as desired.
  
  It remains to prove $\bar u = e$, where $\bar u = \lim_{n \to \infty} u_n$.
  Put
  $$
  v_n = \sum_{k = 0}^n \frac{1}{k!} 
  $$
  for each $n \in \bN$.
  Let $m$, $n \in \bN$ be such that $m \le n$.
  The binomial theorem yields
  $$
  u_n
  = \sum_{k = 0}^n  \binom{n}{k} \frac{1}{n^k}
  = \sum_{k = 0}^n \frac{1}{k!} \cdot \frac{n!}{n^k {(n - k)}!} \, .
  $$
  Besides, $n!$ is less than or equal to $n^k {(n - k)}!$ for each $k \in \{ 0, 1, 2, \dotsc, n \}$.
  It follows $ u_n \le v_n$,  and consequently, $\bar u \le e$.
  The final step is to prove the other inequality.
 For each fixed $k \in \{ 0, 1, 2, \dotsc, n \}$, 
  $$
  \frac{n!}{{(n - k)}! n^k} = \prod_{j = 0}^{k - 1} \left( 1 - \frac{j}{n} \right) 
 $$
 approaches $1$ as $n$ approaches~$\infty$.
 Therefore, by letting $m$ be fixed and $n$ approach $\infty$ in 
 $$
  \sum_{k = 0}^m \frac{1}{k!} \cdot \frac{n!}{n^k {(n - k)}!} \le u_n \,, 
  $$
  we obtain $v_m \le \bar u$, and consequently, $e \le \bar u$.
\end{proof}


Let us briefly comment the proof of Lemma~\ref{lem:convergence-to-e}.
Our proof of the fact that $\left( \left( 1 + n^{-1} \right)^n  \right)_{n \in \bN}$
is increasing is taken from \cite{Wiener85}.
Another approach \cite{GiaquintaModicaApprox} is to closely examine the right-hand side of equality 
$$ 
\left(1 + \frac{1}{n} \right)^n = \sum_{k = 0}^n \frac{1}{k!} \prod_{j = 0}^{k - 1} \left(1 - \frac{j}{n} \right) 
$$
which holds true for every $n \in \bN$.
Equation~\eqref{eq:Bernoulli} is usually called \emph{Bernoulli's inequality};
it holds true for every $n \in \bN$ and every
$x \in \left[- 1, 1 \right[ \cup \left]1, \infty \right[$ \cite{MitrinovicAI}.
Our proof of the fact that
$\left( \left( 1 + n^{-1} \right)^n  \right)_{n \in \bN}$ converges to $e$ is taken from \cite{RudinPrinciples}.


\begin{proof}[Proof of Theorem~\ref{thm:factnn-e}]
  Straightforward computations yield
  $$
  \frac{u_{n + 1}^{n + 1}}{u_n^n} 
  = \left(1 + \frac{1}{n} \right)^n  
  $$
  for every $n \in \bN$,
  so Lemma~\ref{lem:convergence-to-e} ensures that 
  $\left( u_{n + 1}^{n + 1} u_n^{- n} \right)_{n \in \bN}$ is (monotonically) increasing and converges to~$e$.
  Besides, we have $u_1 = 1 < \sqrt{2} = u_2$.
  Therefore, the desired result follows from Lemma~\ref{lem:root-vs-ratio}.
\end{proof} 

\bibliographystyle{plain}
\bibliography{converge}

\end{document}
  
  